\documentclass[a4paper]{article}
\usepackage{amssymb}

\usepackage{amsthm}
\newtheorem{theorem}{Theorem}[section]
\newtheorem{lemma}{Lemma}[section]
\newtheorem{proposition}{Proposition}[section]

\usepackage{amsmath}
\usepackage{graphicx}
\usepackage{float}

\title{Geometric Basic}
\author{listenzcc}

\begin{document}

\maketitle

\abstract
Not done yet.

\tableofcontents

\section{Prepare knowledge}
\subsection{Space}
A \emph{space} is a collection $\mathfrak{C}$ meeting following conditions:

\begin{equation}
    \begin{aligned}
        f(\mathbf{a}, \mathbf{b})                  & \in \mathfrak{C} \quad \forall \mathbf{a}, \mathbf{b} \in \mathfrak{C} \\
        f(\mathbf{a}, \mathbf{o})                  & = \mathbf{a} \quad \exists \mathbf{o} \in \mathfrak{C}                 \\
        f(a \cdot \mathbf{a} + b \cdot \mathbf{b}) & = a f(\mathbf{a}) + b f(\mathbf{b})
    \end{aligned}
\end{equation}

The 3-D space is a classic space, each node in 3-D space can be expressed as $\vec{x} = (x_{1}, x_{2}, x_{3}) \in \mathfrak{R}^{3} $ with a set of axes $(\vec{e}_{1}, \vec{e}_{2}, \vec{e}_{3})$.
To make sure the system can represent every node in the 3-D space, it requires that $det[\vec{e}_{1}, \vec{e}_{2}, \vec{e}_{3}] \neq 0$.

\subsection{Scalar and vector field}
A function of space is known as a field.
Let an arbitrary 3-D coordinate system be given.

If to each position $\vec{x} = (x_{1}, x_{2}, x_{3})$ of a region in space, it corresponds a scalar $\phi (x_{1}, x_{2}, x_{3})$, then $\phi$ is called a \emph{scalar field}.
Like \emph{density} field.

If to each position $\vec{x} = (x_{1}, x_{2}, x_{3})$ of a region in space, it corresponds a vector $\vec{a} (x_{1}, x_{2}, x_{3})$, then $\vec{a}$ is called a \emph{vector field}.
Like \emph{velocity} field.

\subsection{Nabla operator $\nabla$}
Partial derivatives is an useful tool to measure the changes along axes.

For scalar field
\begin{equation}
    \partial_{i} \phi = \frac{\partial}{\partial x_{i}} \phi
\end{equation}

For vector field
\begin{equation}
    (\partial_{i} \vec{a}) (\vec{r}) = \lim_{\Delta x_{i} \rightarrow 0} \frac{\vec{a}(\vec{r} + \Delta x_{i} \vec{e}_{i}) - \vec{a}(\vec{r})}{\Delta x_{i}}
\end{equation}

To simplify the partial derivatives, we imply \emph{Nabla operator}.
\begin{equation}
    \nabla (\cdot) = \sum_{i} \vec{e}_{i} \partial_{i} (\cdot)
\end{equation}

\subsection{Gradient, Divergence and Curl}
Gradient, Divergence and Curl are three basic and important measurement of spatial changes of the field.
\paragraph{Gradient}
\begin{equation}
    grad \phi = \nabla \phi = \sum_{i} \vec{e}_{i} \partial_{i} \phi
\end{equation}

\paragraph{Divergence}
\begin{equation}
    div \vec{a} = \nabla \cdot \vec{a} = \sum_{i} \partial_{i} \vec{a}
\end{equation}

\paragraph{Curl}
\begin{equation}
    curl \vec{a} = \nabla \times \vec{a} = det
    \begin{pmatrix}
        \vec{e}_{1}  & \vec{e}_{1}  & \vec{e}_{1}  \\
        \partial_{1} & \partial_{1} & \partial_{1} \\
        \vec{a}_{1}  & \vec{a}_{1}  & \vec{a}_{1}
    \end{pmatrix}
\end{equation}

\begin{theorem}
    Gradient, Divergence and Curl following the equations.
    \begin{equation}
        \begin{aligned}
            \nabla (\phi + \psi)                  & = \nabla \phi + \nabla \psi                                                     \\
            \nabla \cdot (\vec{a} + \vec{b})      & = \nabla \cdot \vec{a} + \nabla \cdot \vec{b}                                   \\
            \nabla \times (\vec{a} + \vec{b})     & = \nabla \times \vec{a} + \nabla \times \vec{b}                                 \\
            \nabla \cdot (\phi \vec{a})           & = \phi \nabla \cdot \vec{a} + \vec{a} \cdot \nabla \phi                         \\
            \nabla \cdot (\vec{a} \times \vec{b}) & = \vec{b} \cdot (\nabla \times \vec{a}) - \vec{a} \cdot (\nabla \times \vec{b}) \\
            \nabla \times (\nabla \times \vec{a}) & = \nabla (\nabla \cdot \vec{a}) - \nabla^{2} \vec{a}                            \\
            \nabla \cdot (\nabla \times \vec{a})  & = 0                                                                             \\
            \nabla \times \nabla \phi             & = 0
        \end{aligned}
    \end{equation}

    The interesting things are the divergence of curl is zero, and the curl of gradient is zero.
\end{theorem}
\end{document}