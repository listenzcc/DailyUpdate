\documentclass{article}
\usepackage{amsmath}
\usepackage{amssymb}
\usepackage{amsthm}
\newtheorem{theorem}{Theorem}[section]
\newtheorem{lemma}{Lemma}[section]


\title{Beta Distribution}
\author{listenzcc}

\begin{document}

\maketitle
\section{What is Beta Distribution}
The \emph{beta function} (also known as \textit{Euler's integral of the first kind}) is important in calculus and analysis due to its close connection to the gamma function, which is itself a generalization of the factorial function.
Many complex integrals can be reduced to expressions involving the beta function.

\subsection{Definition}
The \emph{beta function}, denoted by \(B(x, y)\), is defined as \[B(x, y) = \int\limits_{0}^{1} t^{x-1} (1-t)^{y-1} dt\]
this is also the Euler's integral of the first kind.

\subsection{Featues}
\paragraph{Symmetry}
Symmetry of the Beta Function \[B(x, y) = B(y, x)\]

\paragraph{Gamma Function}
We have \[B(x, y) = \frac{\Gamma(x)\Gamma(y)}{\Gamma(x+y)}\]
thus for positive integrals $x$ and $y$, we can define the \emph{Beta function} as  \[B(x, y) = \frac{(x-1)!(y-1)!}{(x+y-1)!}\]

\begin{proof}
    proof.
\end{proof}

\begin{theorem}
    theorem.
\end{theorem}

\begin{lemma}
    lemma.
\end{lemma}

\end{document}