\documentclass{beamer}

\usepackage[utf8]{inputenc}

\usetheme{Warsaw}


\title{Maxwell Equations}
\author{listenzcc}
\institute{NICA}
\date{\today}


\begin{document}

\frame{\titlepage}

\begin{frame}
    \frametitle{Table of Contents}
    \tableofcontents
\end{frame}

\section{Prepare Knowledge}

\begin{frame}
    \frametitle{Scalar and vector field}
    A function of space is known as a field.
    Let an arbitrary 3-D coordinate system be given.

    \begin{block}{Scalar field}
        If to each position $x = (x_{1}, x_{2}, x_{3})$ of a region in space, it corresponds a scalar $\phi (x_{1}, x_{2}, x_{3})$, then $\phi$ is called a \emph{scalar field}.

        Like \emph{density} field.
    \end{block}

    \begin{block}{Vector field}
        If to each position $x = (x_{1}, x_{2}, x_{3})$ of a region in space, it corresponds a vector $\vec{a} (x_{1}, x_{2}, x_{3})$, then $\vec{a}$ is called a \emph{vector field}.

        Like \emph{velocity} field.
    \end{block}

    \pause
    How does scalar or vector field change along spatial dimensions?
\end{frame}

\begin{frame}
    \frametitle{Nabla operator $\nabla$}
    Partial derivatives is an useful tool to measure the changes along spatial dimensions.

    For scalar field
    \begin{equation}
        \partial_{i} \phi = \frac{\partial}{\partial x_{i}} \phi
    \end{equation}

    For vector field
    \begin{equation}
        (\partial_{i} \vec{a}) (\vec{r}) = \lim_{\Delta x_{i} \rightarrow 0} \frac{\vec{a}(\vec{r} + \Delta x_{i} \vec{e}_{i}) - \vec{a}(\vec{r})}{\Delta x_{i}}
    \end{equation}

    To simplify the partial derivatives, we imply \emph{Nabla operator}.
    \begin{alertblock}{Nabla operator}
    \begin{equation}
        \nabla (\cdot) = \sum_{i} \vec{e}_{i} \partial_{i} (\cdot)
    \end{equation}
    \end{alertblock}

\end{frame}

\begin{frame}
    \frametitle{Gradient, diverence, curl}

    \begin{block}{Gradient}
        \begin{equation}
            grad \phi = \nabla \phi = \sum_{i} \vec{e}_{i} \partial_{i} \phi
        \end{equation}
    \end{block}

    \begin{block}{Divergence}
        \begin{equation}
            (div \vec{a}) = (\nabla \vec{a}) = \sum_{i} (\partial_{i} \vec{a})
        \end{equation}
    \end{block}

    \begin{block}{Curl}
        \begin{equation}
            curl \vec{a} = \nabla \times \vec{a} = det
            \begin{pmatrix}
                \vec{e}_{1}  & \vec{e}_{1}  & \vec{e}_{1}  \\
                \partial_{1} & \partial_{1} & \partial_{1} \\
                \vec{a}_{1}  & \vec{a}_{1}  & \vec{a}_{1}
            \end{pmatrix}
        \end{equation}
    \end{block}

\end{frame}
\section{Maxwell Equations}

\begin{frame}
    \frametitle{Definition}
\end{frame}

\section{Forward and Backward Process}

\begin{frame}
    \frametitle{Forward process}
\end{frame}

\begin{frame}
    \frametitle{Backward process}
\end{frame}

\section{Source Location}

\begin{frame}
    \frametitle{Formulation}
\end{frame}

\begin{frame}
    \frametitle{Solution}
\end{frame}

\section{Slide Samples}

\begin{frame}
    \frametitle{Items}
    \begin{itemize}
        \item<1-> Text visible on slide 1
        \item<2-> Text visible on slide 2
        \item<3> Text visible on slide 3
        \item<4-> Text visible on slide 4
    \end{itemize}
\end{frame}

\begin{frame}
    \frametitle{Pause}
    In this slide \pause
    the text will be partially visible \pause
    And finally everything will be there
\end{frame}

\begin{frame}
    \frametitle{Sample frame title}

    In this slide, some important text will be
    \alert{highlighted} because it's important.
    Please, don't abuse it.

    \begin{block}{Remark}
        Sample text
    \end{block}

    \begin{alertblock}{Important theorem}
        Sample text in red box
    \end{alertblock}

    \begin{examples}
        Sample text in green box. The title of the block is ``Examples".
    \end{examples}
\end{frame}


\begin{frame}
    \frametitle{Two-column slide}

    \begin{columns}

        \column{0.5\textwidth}
        This is a text in first column.
        $$E=mc^2$$
        \begin{itemize}
            \item First item
            \item Second item
        \end{itemize}

        \column{0.5\textwidth}
        This text will be in the second column
        and on a second through this is a nice looking
        layout in some cases.

    \end{columns}
\end{frame}

\end{document}